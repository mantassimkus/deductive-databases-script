\documentclass[12pt]{extarticle}


 %
 
\usepackage[%
  papersize={16cm,9cm},
  hmargin=1cm,%
  vmargin=1cm,%
  head=0.5cm,% will be changed later
  headsep=0pt,%
  foot=0.5cm% will be changed later
  ]{geometry}

  \pagestyle{empty}

  \newcommand{\nextslide}{\vfill \hfill $\downarrow$ \newpage}

  \newcommand{\stresspoint}{\newline $\bullet$\quad }

%%% Local Variables:
%%% mode: latex
%%% TeX-master: "main"
%%% End:


\usepackage{fullpage}


  \newcommand{\nextslide}{}

  \newcommand{\stresspoint}{}


%%% Local Variables:
%%% mode: latex
%%% TeX-master: "main"
%%% End:

  
%
\usepackage{times}
\usepackage{amsmath}
\usepackage{amsthm}
\usepackage{amssymb}

\newtheorem{definition}{Definition}
\newtheorem{proposition}{Proposition}
\newtheorem{example}{Example}

\begin{document}


\section*{Preface}

This is the preface

%%% Local Variables:
%%% mode: latex
%%% TeX-master: "main"
%%% End:



\tableofcontents

\section{Introduction (Road Map)}


\cleardoublepage

\section{Propositional Logic and Basic Notation}

\newcommand{\allprops}{\Sigma_{prop}}

We recall the basics of \emph{Propositional Logic (PL)}.



\begin{definition}[Propositional atoms]
  We assume a (countable, infinite) set $\allprops$  of
  \emph{propositional atoms} (we usually use $p,q,r,q',p_1,p_2$ and
  similar sybmolsfor propositional atoms).
\end{definition}

\begin{definition}[Syntax of PL]
  We assume a (countable, infinite) set $\allprops$  of
  \emph{propositional atoms} (we usually use $p,q,r,q',p_1,p_2$ and
  similar sybmolsfor propositional atoms).  The definition of \emph{formulas} in PL  uses induction:
  \begin{enumerate}
  \item Every propositional atom $p\in \allprops$ is a formula.
  \item The symbols $\top$ and $\bot$ (called ``top'' and ``bottom'', respectively) are formulas.
   \item If $\varphi$ and $\psi$ are formulas, then also expressions $(\neg \varphi)$, $(\varphi\lor \psi)$, $(\varphi\land \psi)$, and $(\varphi\supset\psi)$ are also formulas.
  \end{enumerate}
\end{definition}



\begin{definition}[Semantics of PL]
  An \emph{interpretation} $m$ (in PL) is a function
  $m:\allprops \rightarrow \{0,1\}$.
\end{definition}
Intuitively, an interpretation $m$ assigns to every propositional atom
$p\in \allprops$ some truth value, where $0$ corresponds to $\mathit{false}$ and $1$ corresponds to $\mathit{true}$.


\begin{definition}
  Assume an intepretation $m$ and a propositional atom $p$. We write
  $m\models p$ if $m(p)=1$. Furhter write:
  \begin{enumerate}
  \item $m \models \varphi \land \psi  $, if $m \models \varphi$ and $m \models \psi$;
  \item $m \models \varphi \lor \psi  $, if $m \models \varphi$ or $m \models \psi$;
  \item $m \models \neg \varphi $, if $m \not\models \varphi$;
  \item $m \models  \varphi \supset \psi$, if $m \models  \varphi $ implies $m \models  \psi$ (in other words, if  $m \models  \psi$ or  $m \not\models  \varphi $;
    
  \end{enumerate}
  Assume a formula $\varphi$ and an interpretaiton $m$. If
  $m\models \varphi$, we say $\varphi$ is \emph{true under} $m$, and
  that $m$ is a\emph{model} of $\varphi$. If $m\not\models \varphi$,
  we say $\varphi$ is \emph{false under} $m$. We say a formula
  $\varphi$ is \emph{satisfiable} if there exists some $m$ such that
  $m\models \varphi$. \stresspoint We say a formula $\varphi$ is \emph{valid} if
  $m\models \varphi$ holds for all interpretations $m$. \stresspoint We use
  $\mathit{mods}(\varphi)$ to denote the set of models of a formula
  $\varphi$, i.e.\,$\mathit{mods}(\varphi)=\{m\mid m\models\varphi\}$.
  A pair of formulas $\varphi,\psi$ are \emph{equivalent} if
  $\mathit{mods}(\varphi)=\mathit{mods}(\psi)$, i.e.\,for all
  interpretations $m$, we have that $m\models \varphi$ if and only if
  $m\models \psi$.
\end{definition}
\nextslide
\begin{definition}[Propositional Theories]
  A set of formulas is called a \emph{(propositional) theory}. An
  interpretation $m$ is called a \emph{model} of a theory $\Phi$, if
  $m\models \varphi$ holds for all $\varphi\in \Phi$. If $m$ is a model of a
  theory $\Phi$, we may write $m\models \Phi$.

  A theory $\Phi$ is called \emph{satisfiable}, if there exists at
  least one interpretation $m$ such that $m\models \Phi$. We say a
  theory $\Phi_1$ is \emph{equivalent} to a theory $\Phi_2$, if they
  have the same models.
\end{definition}
Note that a theory $\Phi$ can be seen as the formula
$\bigwedge_{\varphi\in\Phi}\varphi$, i.e., $\Phi$ and
$\bigwedge_{\varphi\in\Phi}\varphi$ have exactly the same models.

\begin{definition}[Normal Forms]
  Formulas in $\{p,\neg p\mid p\in \allprops\}$ are called
  \emph{literals}, i.e.,\,a literal is a either a propositional atom
  or the negation of a propositionasl atom. A literal of the form
  $\neg p$ is said to be \emph{negative}, while literals of the form
  $p$ are \emph{positive}.

  A \emph{clause} is a formula of the form
  $\ell_1\lor \cdots\lor \ell_n$, where $\ell_1,\ldots,\ell_n$ are
  literals. For a clause $C$, let $C^{+}$ denote the positive literals
  that appear in $C$. For a clause $C$, let $C^{-}$ denote the
  negative literals that appear in $C$.

  A formula $\varphi$ is in \emph{conjunctive normal form (CNF)} if it
  has the form $C_1\land \cdots \land C_m$, where $C_1,\ldots,C_m$ are
  clauses.

  A formula $\varphi$ is called \emph{positive}, if the negation
  $\neg$ does not occur in $\varphi$.

  A clause $C$ is called a \emph{Horn clause}, if $|C^{+}|\leq 1$,
  i.e.\,$C$ contains at most one positive literal.

  A clause $C$ is called \emph{definite}, if $|C^{+}|= 1$, i.e.\,$C$
  contains exactly one one positive literal.

  A formula $\varphi$ in CNF is called a \emph{Horn formula} if all
  its clauses are Horn clauses.  A formula $\varphi$ in CNF is called
  a \emph{definite formula} if all its clauses are definite clauses.
\end{definition}

\begin{proposition}
  For any formula $\varphi$, one can build a formula $\varphi'$ such
  that $\varphi$ and $\varphi'$ are equivalent.
\end{proposition}
\begin{proof}
  Use the usual transformations.
\end{proof}


\begin{proposition}
  Every definite formula $\varphi$ is satisfiable.
\end{proposition}
\begin{proof}
  Assume a definite formula $\varphi$. 
\end{proof}
 
\subsection{Notation / Model Minimality}


Observe that an interpretation $m$ can be \emph{represented} as a set
$I\subseteq \allprops$.
\begin{definition}
Given an intepretation $m$, we let
$I^{m}=\{p\in\allprops \mid m(p)=1\}$. 
\end{definition}
That is, $I^{m}$ is the set of
atoms that are made true in $m$.
\begin{definition}
Given a set $I\subseteq \allprops$,
we define the interpretation $m_I$ such that: (i) for all $p\in I$,
$m_{I}(p)=1$, and (ii) for all $p\in \allprops \setminus I$,
$m_{I}(p)=0$.
\end{definition}
Thus from now on, a set $I\subseteq \allprops$ is also
seen as an interpretation, and the previously defined notions are
naturally extended. E.g., $I\models \varphi$ means
$m_{I}\models \varphi$ and so on.

\begin{definition}
  We say $I\subseteq \allprops$ is a \emph{minimal model} of a formula
  $\varphi$, if (i) $I$ is a model of $\varphi$, and (ii) there exists
  no $J\subsetneq I$ such that $J$ is a model of $\varphi$.
\end{definition}
\begin{definition}
We say $I\subseteq \allprops$ is a \emph{maximal model} of a formula
  $\varphi$, if (i) $I$ is a model of $\varphi$, and (ii) there exists
  no $J\supsetneq I$ such that $J$ is a model of $\varphi$.
\end{definition}

\begin{example}
  Consider the formula $p\supset q$. It has 3 models, which are $I_1=\emptyset$,
  $I_2=\{q\}$, $I_3=\{p,q\}$. Then $I_1$ is a minimal model, $I_3$ is a maximal model. 
\end{example}
Note that a formula might have multiple minimal and maximal models.
\begin{example}
  Consider the formula $\varphi = (p \lor q) \land (\neg p\lor \neg q)$. It has 2 models $I_1=\{p\}$ and $I_2=\{p\}$. Observe that $I_1$ is simulaniously a minimal and a maximal model of the formula $\varphi$.
  The same holds for $I_2$.
\end{example}
Note that a formula can have exponentially many minimal models:
\begin{example}
  Consider the formula
  $\varphi = (p_1 \lor q)_1 \land \cdots \land (\neg p_n\lor \neg
  q_n)$. Each model $I$ of $\varphi$ can be obtained as the union $P\cup Q$, where
  $P\subseteq\{ p_1,\ldots,p_n\}$ is arbitrary, and $Q = \{q_i\mid p_i\not\in P, 1\leq i \leq n\}$.
\end{example}
Some formulas have a unique minimal model.
\begin{example}
  Consider the formula $\varphi= \mathit{Rain} \land (\mathit{Rain} \supset \mathit{GrassWet})\land (\mathit{GrassWet} \supset \mathit{GrassSlipery}) $.
 This formula has a unique minimal model $I = \{ \mathit{Rain}, \mathit{GrassWet},\mathit{GrassSlipery}\}$. 
\end{example}


\cleardoublepage

% \section{Basics of Computational Complexity}

% \subsection{Turing Machine}

% \subsection{Complexity Class}

% \subsection{Reductions / Completeness}

% \subsection{Syntax}

% \subsection{Semantics}

% \subsection{Examples}

\section{Propositional Horn Programs}

\subsection{Syntax}

\subsection{Semantics}

\subsection{Complexity}

\subsection{Running on grid: Turing Machine Encoding}

\section{Datalog}

\subsection{Syntax}

\subsection{Semantics}

\subsection{Complexity}

\subsection{Data complexity}

\subsection{Monotonicity}

\section{Answer Set Programming (Normal Programs)}


\subsection{Syntax}

\subsection{Semantics}

\subsection{Complexity}

% \section{Quantified Boolean Formulas}

% \subsection{Syntax}

% \subsection{Semantics}

% \subsection{Encoding of ASP}

\section{Answer Set Programming (Disjunctive Programs)}

\subsection{Syntax}

\subsection{Semantics}

\subsection{Complexity}



\section{Non-ground ASP}

\section{Strong Equivalence}

\section{Well-founded semantics for logic programs}


\section{Towards DLs: Open-world vs Closed-world}

\section{Description Logics: Syntax and Semantics}

\section{Description Logics: Reasoning Tasks}

\section{Description Logics with Closed Predicates}

\subsection{Motivation}

\subsection{Syntax}

\subsection{Semantics}

\subsection{Encoding Closed Predicates using Nominals}

\subsection{Discussion of Non-monotonicity}

\section{Clopen Knowledge Bases}

\subsection{Motivation}

\subsection{Syntax}

\subsection{Semantics}

\subsection{Examples}

\subsection{Discussion of Computational Complexity}


\section{Guarded Existential Rules}

\subsection{Motivation}

\subsection{Syntax}

\subsection{Semantics}

\subsection{Examples}

\subsection{Discussion of Computational Complexity and Expressive Power}


\section{DL programs}

\subsection{Motivation}

\subsection{Syntax}

\subsection{Semantics}

\subsection{Examples}

\subsection{Discussion of Computational Complexity}

\section{SHACL}
\label{sec:shacl}

\subsection{Motivation}

\subsection{Syntax}

\subsection{Intuitive Semantics}

\subsection{Semantics by Translation into ASP}

\subsection{Examples}

\end{document}



%%% Local Variables:
%%% mode: latex
%%% TeX-master: t
%%% End:
